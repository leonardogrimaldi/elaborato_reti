\documentclass[a4paper,12pt]{report}
\usepackage{alltt, fancyvrb, url}
\usepackage{graphicx}
\usepackage[utf8]{inputenc}
\usepackage{float}
\usepackage{hyperref}
\usepackage{listings}
\usepackage{color}
\usepackage{minted}

% Questo commentalo se vuoi scrivere in inglese.
\usepackage[italian]{babel}

\usepackage[italian]{cleveref}
\title{Relazione del progetto di Programmazione di Reti 
    \\ Traccia 3: Monitoraggio di Rete}

\author{Leonardo Grimaldi}
\date{\today}   
\begin{document}
\maketitle
\tableofcontents
\chapter{Consegna}
Realizzare uno script Python per monitorare lo stato di una rete, controllando la disponibilità di uno o più host tramite il protocollo ICMP (ping).
%
Lo script deve consentire all'utente di specificare gli indirizzi IP degli host da monitorare e deve visualizzare lo stato (online/offline) di ciascun host.
\chapter{Funzionamento}
All'avvio dello script \texttt{client.py} verrà chiesto all'utente di inserire in console l'hostname (es. \texttt{google.com}) oppure l'indirizzo IP (es. \texttt{8.8.8.8}) della macchina della quale si vuole sapere la disponibilità.
%
Qualora arrivi la risposta del destinatario, si otterrà un messaggio del formato: \texttt{17 byte da 216.58.204.238: icmp\_seq=0 ttl=111 tempo=16 ms}, dove il numero di byte è dato dalla dimensione della sezione DATA (nel nostro caso \mintinline{python3}|DATA = "Buongiorno mondo!"|) e il tempo indica il delay, ovvero quanto ci ha messo a ricevere la risposta dal tempo di invio.
%
Nel caso in cui non si riceva una risposta, a schermo verrà solamente stampato quanti byte sono stati inviati, oppure in casi eccezionali un messaggio di errore.
%
Per terminare correttamente lo script su \texttt{Windows} bisogna utilizzare la combinazione di tasti \texttt{CTRL + Break}.
\chapter{Codice}
Il sorgente è composto da tre sezioni principali necessarie per l'implementazione base del \texttt{ping}, che andrò ad elencare e spiegare in dettaglio.
\section{\texttt{ICMPchecksum(packet)}}
Il codice scritto in questo metodo contiene la logica di calcolo di un checksum ICMP.
%
Prende in input un oggetto \texttt{bytes()} e restituisce in output un \texttt{int} con soli 16 bit meno significativi utili.
%
\begin{minted}[linenos]{python3}
  def ICMPchecksum(packet):
  temp = packet
  if len(temp) % 2 != 0:
      temp += bytes([0])
\end{minted}
Il pacchetto viene spostato in una variabile di appoggio \texttt{temp} per evitare di modificare il parametro in input e causare inconsistenze nel codice.
%
\\ Successivamente viene eseguito un controllo sulla parità: nel caso in cui il numero di byte del pacchetto è dispari, viene aggiunto alla fine un byte \texttt{x00} di padding.
\begin{minted}[linenos, firstnumber=last]{python3}
  first = int.from_bytes(temp[0:2], byteorder='big')
  sum = first
  for i in range(2, len(temp) - 1, 2):
      next = int.from_bytes(temp[i:i+2],byteorder='big')
      sum += next
\end{minted}
In questa porzione di codice vengono presi 2 byte e convertiti in \texttt{int} e sommati con i due successivi finché non si raggiunge la fine.
\begin{minted}[linenos, firstnumber=last]{python3}
  overflow = sum >> 16
  checksum = ~(sum + overflow) & 0xFFFF
  return checksum
\end{minted}
Si identifica l'\texttt{overflow} prendendo i 16 bit più significativi e lo si aggiunge a \texttt{sum}. Si effettua il complemento a 1 di tale valore con il carattere \texttt{\~} e poi si azzerano i primi 16 bit (ridondanti dato che il checksum è a 16 bit mentre l'intero a 32).
%
\\ Notare che \textbf{il metodo \texttt{ICMPchecksum} non azzera in alcun modo il campo `checksum' di un pacchetto}, sarà infatti compito del chiamante farlo.
\section{\texttt{ping(mySocket, destinationHost,\\ identifier, sequenceNumber)}}
Viene usato per effettuare un messaggio di Echo request a un destinatario.
Riceve in input:
\begin{itemize} 
  \item \texttt{mySocket}: oggetto \texttt{socket} da usare per l'invio del pacchetto
  \item \texttt{destinationHost}: indirizzo IP oppure host name del destinatario
  \item \texttt{identifier}: identificatore del pacchetto da inviare
  \item \texttt{sequenceNumber}: numero di sequenza del pacchetto
\end{itemize}
\begin{minted}[linenos, breaklines, breakbytoken]{python3}
  def ping(mySocket, destinationHost, identifier, sequenceNumber):
  checksum = 0
  header = struct.pack('!BBHHH', TYPE_ECHO_REQUEST, CODE_ECHO_REQUEST, checksum, identifier, sequenceNumber)
  # len(data) Ci dà il numero di caratteri nella stringa
  # esempio: len(data) = 10
  # struct.pack('!10s', ...) dice quindi di usare 10 byte
  # encode() usa l'encoding UTF-8 di default. 
  data = struct.pack('!' + str(len(DATA)) + 's', DATA.encode())
  packet = header + data
\end{minted}
Prima di inviare il pacchetto è necessario costruirlo: usando la funzione \texttt{struct.pack()} è possibile convertire le variabili di Python in oggetti \texttt{bytes()} in modo da manipolarli più facilmente.
%
Passando ad essa la format string \texttt{`!BBHHH'} si può indicare come verranno rappresentati i dati letti nei successivi parametri:
\begin{itemize}
  \item \texttt{!}: ordine byte network (big endian)
  \item \texttt{B}: unsigned char (1 byte)
  \item \texttt{H}: unsigned short (2 byte)
\end{itemize}
Elencando successivamente le variabili che formano il pacchetto ICMP (ricordando che il checksum deve essere posto a \texttt{x00} prima che venga calcolato), è possibile ottenere una rappresentazione in \texttt{byte()} del header.
%
Alla riga \texttt{9} viene formato il pacchetto accodando i byte di data su quelli di header.
\begin{minted}[linenos, firstnumber=last, breaklines, breakbytoken]{python3}
  chk = ICMPchecksum(packet)
  header = struct.pack('!BBHHH', TYPE_ECHO_REQUEST, CODE_ECHO_REQUEST, chk, identifier, sequenceNumber)
  packet = header + data
\end{minted}
In questa porzione si calcola il checksum del pacchetto e lo si ricostruisce inserendone il valore ottenuto.
\begin{minted}[linenos, firstnumber=last, breaklines, breakbytoken]{python3}
  sentTime = time.time()
  bytesSent = mySocket.sendto(packet, (destIP, 1))
  return bytesSent, sentTime
\end{minted}
Infine invio il pacchetto usando \texttt{socket.sendto()} e restituisco il tempo di invio e il numero di byte inviati
\section{\texttt{receive\_reply(mySocket, myID, timeout)}}
Questo metodo gestisce l'arrivo del pacchetto di risposta ICMP (Echo reply).
%
In input riceve l'oggetto \texttt{socket} su cui leggere i pacchetti, l'identificatore \texttt{myID} per verificare che il pacchetto sia nostro e il tempo di attesa \texttt{timeout}.
%
\begin{minted}[linenos, breaklines]{python3}
  def receive_reply(mySocket, myID, timeout):
  timeLeft = timeout
  while True:
      startedSelect = time.time()
      # aspetto che il 'mySocket' sia in stato read ovvero ricezione pachetti
      readable, writeable, exceptional = select.select([mySocket], [], [], timeout)
      selectTime = (time.time() - startedSelect)
\end{minted}
Creo un ciclo `infinito' che dovrà ovviamente fermarsi quando è scaduto il mio \texttt{timeout}.
%
Aspetto che il socket sia in modalità lettura e in \texttt{selectTime} inserisco il tempo che ci ha messo ad attendere.
\begin{minted}[linenos, firstnumber=last, breaklines]{python3}
  # select timeout
  if not (readable or writeable or exceptional):
      return None, None, None, None, None 
  timeReceived = time.time()
  packet, address = mySocket.recvfrom(ICMP_MAX_RECV)
\end{minted}
Gestisco il caso di non leggibilità e continuo estraendo il pacchetto ricevuto con \texttt{socket.recvfrom()}.
%
Il parametro \texttt{ICMP\_MAX\_RECV} indica la dimensione del buffer, in questo caso \texttt{2048} bytes, oltre i quali il dato verrà spezzato.
\begin{minted}[linenos, firstnumber=last, breaklines]{python3}
  ipHeader = packet[:20]
  (iphVersion, iphTypeOfSvc, iphLength, iphID, iphFlags, iphTTL,
   iphProtocol, iphChecksum, iphSrcIP, iphDestIP) = struct.unpack("!BBHHHBBHII", ipHeader)
  icmpHeader = packet[20:28]
  icmpType, icmpCode, icmpChecksum, \
  icmpPacketID, icmpSeqNumber = struct.unpack("!BBHHH", icmpHeader)
\end{minted}
Usando \texttt{struct.unpack()} converto i \texttt{bytes()} del \texttt{packet} in variabili Python che potrò tornare e visualizzare più facilmente.
%
\begin{minted}[linenos, firstnumber=last, breaklines]{python3}
  if icmpPacketID == myID: # Nostro pacchetto
    dataSize = len(packet) - 28
    return timeReceived, dataSize, iphSrcIP, icmpSeqNumber, iphTTL
  timeLeft = timeLeft - selectTime
  if timeLeft <= 0:
    return None, None, None, None, None
\end{minted}
Confronto l'ID del pacchetto con il mio (quello per cui devo ricevere la risposta): se la condizione è vera ritorno alcune informazioni relative ad esso.
%
Altrimenti, riciclo lo stesso procedimento se non ho ancora superato il tempo \texttt{timeout}.
\end{document}